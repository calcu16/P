\section{Shallow Copying}
Instead of always doing deep copies each time an array is modified, we only need to do shallow copies and reference count for the same effect.
\subsection{Exclusive update}
If there is only one reference to a value, and that value is being updated via that reference, then a shallow copy is not necessary because the original would be automatically garbage collected.
    Therefore the update can be done in place.
\subsection{Repeated array update}
With shallow copying, if an array is being update within a section of code repeatedly, there will only be one instance of the array after the first copy.
    Therefore it is safe to the latter array updates without checking to see if the reference is exclusive.

\section{Aliased Function Arguments}
Assuming the write through\/back function calling : 

Currently function arguments can still be midly aliased.
    This is because the result writes back to the original data.
    However we can do static analysis, and determine that if a given set of arguments are aliased, for all function calls beneath it, what arguments are aliased.
    Thus starting at main, we can simplify many functions whose arguments are not aliased, and generate seperate code for them.

\subsection{Write-Only Function Arguments}
Some arguments are only going to be written to.
    Several optimizations can be made with respect to aliasing.
    First if variables that are not read from are aliased, then we can ignore those aliasing.
    If a write-only argument is aliased with a read or read/write argument we do not have to generate an extra copy of the write only argument as that will never be read from.

